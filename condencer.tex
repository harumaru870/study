\documentclass[10pt,dvipdfmx]{article}[b5paper]
\usepackage[utf8]{inputenc}
\usepackage{amsmath}
\usepackage{amssymb}
\usepackage{enumerate}
\usepackage{xcolor}
\usepackage{tcolorbox}
\tcbuselibrary{breakable,theorems,skins}
\usepackage{geometry}
\usepackage{fancyhdr} % ヘッダーとフッターをカスタマイズするためのパッケージ
\usepackage{graphicx}


\definecolor{safecolor}{rgb}{0.2, 0.4, 0.8}

\newtcolorbox{kousiki}[2][]%%%〇〇の定理
{enhanced,%%%tikzを用いた記法の処理
left=22pt,right=22pt,%%%box内左右の余白
fonttitle=\gtfamily\bfseries\large,%%%タイトルのフォント指定
coltitle=white,%%%タイトルの文字の色
colbacktitle=blue!50!black,%%%タイトルの背景の色
attach boxed title to top left={},%%%タイトルを左寄せに、少し微調整
boxed title style={skin=enhancedfirst jigsaw,arc=1mm,bottom=0mm,boxrule=0mm},%%%タイトルボックスの装飾
boxrule=0.5pt,%%%枠線の太さ
colback=safecolor!5!,%%%本文の背景色
colframe=safecolor,%%%本文の枠の色
sharp corners=northwest,%%%左上の角の調整 
drop fuzzy shadow,%%%影をつける
breakable,%%%ページマタギOK
title=\vspace{3mm}#2,%%%タイトルは直接入力
arc=1mm,%弧
fontupper=\gtfamily,%本文のフォントを太く(数式は除く)
#1}%


\geometry{top=25mm, headheight=15mm} % ヘッダーに十分なスペースを確保

\pagestyle{fancy}
\fancyhf{} % 既存のヘッダーとフッターをクリア
\fancyhead[C]{コンデンサーの過渡現象(山岸用)} % ヘッダーの中央にテキストを配置
\renewcommand{\headrulewidth}{2pt} % ヘッダーの下の線の太さ
\renewcommand{\footrulewidth}{0pt} % フッターの線は表示しない

\begin{document}
\includegraphics[width=13cm]{RC直列回路-600x352.jpg}

上図は抵抗$R$[Ω]、コンデンサ$C$[F]、直流電源$E$[V]、スイッチSWからなるRC直列回路です。

\begin{kousiki}{条件}
\vspace{2mm}
 \begin{enumerate}
        \item スイッチSWをオンにした時の時間$t$を$t=0$[s]とする。
\vspace{2mm}
        \item スイッチSWをオンにする前にコンデンサ$C$に蓄えられている電荷$q(t)$はゼロ。\\\[q(0)=0\]である。
    \end{enumerate}
\end{kousiki}
スイッチSWをオンにすると、過渡現象が生じる。
\begin{tcolorbox}
 \begin{enumerate}
        \item 電流$i(t)$が流れて、コンデンサ$C$に蓄えられる電荷$q(t)$が増加し、電圧$Vc(t)$が上昇する。
\vspace{2mm}
        \item 十分に時間がたった時、電流$i(t)$が流れなくなる(0[A])。また、その時のコンデンサは開放されている状態となり、電圧$Vc(t)$が電源電圧$E$と等しくなる。
\end{enumerate}
\end{tcolorbox}
この時、電流$i(t)$が一定値0[A]となった状態を「定常状態」、それ以前の状態を「過渡状態」といい、その過程で見られる現象を「過渡現象」とよぶ。

RC直列回路に流れる電流$i(t)$とコンデンサに蓄えられる電荷$Q(t)$の関係は次式。
\begin{tcolorbox}
\[i(t)=\dfrac{dq(t)}{dt}\]
\end{tcolorbox}
また、RC直列回路にキルヒホッフ第二法則(電圧則)を用いると以下のようになる。
\begin{tcolorbox}
\[E =v_r(t)+v_c(t)\]
\end{tcolorbox}
また
\begin{align*}
 &v_r(t)-R_i(t)=R\dfrac{dq(t)}{dt}\\
 &v_c(t) =\dfrac{1}{C}\int i(t)dt =\dfrac{1}{C}\int (\dfrac{dq(t)}{dt})dt =\dfrac{q(t)}{C}\\
\text{を代入すると、}\\
E& = v_R(t) + v_C(t)\\
&=R\dfrac{dq(t)}{dt} +\dfrac{q(t)}{C}
\text{を得る。}\\
R\dfrac{dq(t)}{dt} &= \dfrac{CE-q(t)}{C}\\
\Leftrightarrow
\dfrac{dq(t)}{dt} &=\dfrac{CE-q(t)}{CR}\\
\dfrac{1}{CE-q(t)}\dfrac{dq(t)}{dt} &=\dfrac{1}{CR}\\
\dfrac{1}{CE-q(t)}dq(t) &= \dfrac{1}{CR}dt\\
\int\dfrac{1}{CE-q(t)}dq(t) &= \int\dfrac{1}{CR}dt\\
\vspace{2mm}
k=CE-q(t)\quad(k\geqq0) \text{と置く。}\\
\text{左辺}&=\int\dfrac{1}{CE-q(t)}dq(t)\\
&=-\int\dfrac{1}{k}dk\\
&=\log_e k +A\\
&=-\log_e(CE-q(t)) +A \quad\text{(Aは積分定数。)}\\
\vspace{2mm}
\text{右辺}&=\int\dfrac{1}{CR}dt\\
&=\dfrac{1}{CR}\int dt\\
&=\dfrac{1}{CR}t+B \quad\text{(Bは積分定数。)}\\
-\log_e(CE-q(t)) +A &= \dfrac{1}{CR}t +B\\
A-B &=D\text{と置く。}\\
\log_e(CE-q(t)) &= -\dfrac{1}{CR}t +D\\
\Leftrightarrow
CE-q(t) &=e^{-\frac{1}{CR}t+D}\\
&=e^{-\frac{1}{CR}t} \times e^D\\
\Leftrightarrow
q(t) &= CE - e^{-\frac{1}{CR}t} \times e^D\\
& \quad\text{ここでDを考える。$t=0$のとき、電荷$q(0)=0$であるので}\\
q(0) &= CE - e^{-\frac{1}{RC}\times 0} \times e^D\\
\Leftrightarrow 0 &= CE - e^0 \times e^D\\
&= CE- 1 \times e^D\\
\therefore e^D &= CE\\
q(t) &= CE - e^{-\frac{1}{CR} t} \times e^D\\
&= CE - e^{-\frac{1}{CR} t} \times CE\\
&= CE (1-e^{-\frac{1}{CR} t})\\
\end{align*}
\begin{align*}
\text{また、}i(t)\text{を考えると}\\
i(t) &= CE \frac{d}{dt} \left(1 - e^{-\frac{1}{CR} t}\right) \\
&= CE \left(\frac{d}{dt} 1 - \frac{d}{dt} e^{-\frac{1}{CR} t}\right) \\
&= CE \left(0 - \left(-\frac{1}{CR} e^{-\frac{1}{CR} t}\right)\right) \\
&= CE \left(\frac{1}{CR} e^{-\frac{1}{CR} t}\right) \\
&= \frac{E}{R} e^{-\frac{1}{CR} t}
\end{align*}
% 計算スペース
\end{document}